\hypertarget{02-rcpp_8cpp-example}{}\section{02-\/rcpp.\+cpp}
Shows how can this library be used within R.


\begin{DoxyCodeInclude}
\textcolor{preprocessor}{#include <Rcpp.h>}
\textcolor{preprocessor}{#include "pruner.h"}  

\textcolor{keyword}{using namespace }Rcpp; 

\textcolor{comment}{// Users work from here on -----------------------------------------------------}

\textcolor{comment}{// Need to declare this class}
\textcolor{keyword}{class }pruner::FunArgs \{
  \textcolor{keywordtype}{int} nnodes;
\textcolor{keyword}{public}:
  
  ~FunArgs() \{\};
  FunArgs(\textcolor{keywordtype}{int} n): nnodes(n) \{\};
  
  \textcolor{keywordtype}{int} get\_nnodes() \{\textcolor{keywordflow}{return} this->nnodes;\};
  \textcolor{keywordtype}{void} set\_nnodes(\textcolor{keywordtype}{int} n) \{
    this->nnodes = n;
    \textcolor{keywordflow}{return};
  \};
\};

\textcolor{keywordtype}{void} myfunction(
    std::shared\_ptr<pruner::FunArgs> a,
    \hyperlink{classpruner_1_1Tree}{pruner::Tree} * t,
    \hyperlink{classpruner_1_1TreeIterator}{pruner::TreeIterator} * iter) \{
  
  \textcolor{comment}{// Moving a single step up}
  printf(\textcolor{stringliteral}{"Currently sitting on the node %i.\(\backslash\)nCurrent parents are: "}, iter->\hyperlink{classpruner_1_1TreeIterator_ae76ec4f2c1390d72889f78a51d65d19c}{id}());
  \textcolor{keywordflow}{for} (pruner::v\_uint::const\_iterator i = iter->begin\_par(); i != iter->end\_par(); ++i) \{
    printf(\textcolor{stringliteral}{" %i"}, *i);
  \}
  printf(\textcolor{stringliteral}{"\(\backslash\)n"});
  
  iter->\hyperlink{classpruner_1_1TreeIterator_adca1d999f093a69e2f5d044b358e5da7}{up}();
  printf(\textcolor{stringliteral}{"Currently sitting on the node %i.\(\backslash\)nCurrent parents are: "}, iter->\hyperlink{classpruner_1_1TreeIterator_ae76ec4f2c1390d72889f78a51d65d19c}{id}());
  \textcolor{keywordflow}{for} (pruner::v\_uint::const\_iterator i = iter->begin\_par(); i != iter->end\_par(); ++i) \{
    printf(\textcolor{stringliteral}{" %i"}, *i);
  \}
  printf(\textcolor{stringliteral}{"\(\backslash\)n"});
  \textcolor{keywordflow}{return};
\}

\textcolor{comment}{// [[Rcpp::export]]}
List fancytree(\textcolor{keyword}{const} pruner::v\_uint & parents, \textcolor{keyword}{const} pruner::v\_uint & offspring) \{
  
  \textcolor{comment}{// Creating the object}
  \hyperlink{typedefs_8h_a91ad9478d81a7aaf2593e8d9c3d06a14}{uint} ans;
  \hyperlink{classpruner_1_1Tree}{pruner::Tree} mytree(parents, offspring, ans);
  
  \textcolor{comment}{// Adding function arguments}
  mytree.\hyperlink{classpruner_1_1Tree_ac205a6f18c76016b2c526e62ed986d24}{args} = std::make\_shared< pruner::FunArgs >(1);
  mytree.\hyperlink{classpruner_1_1Tree_a6a54846f5e95a7d07369e4075ff651a7}{fun} = myfunction;
  
  \textcolor{comment}{// Calling functions}
  mytree.\hyperlink{classpruner_1_1Tree_ac205a6f18c76016b2c526e62ed986d24}{args}->set\_nnodes(5);
  mytree.eval\_fun();   \textcolor{comment}{// Implicit call}
  
  \textcolor{comment}{// Example printing}
  mytree.print();
  
  \textcolor{keywordflow}{return} List::create(
    \_[\textcolor{stringliteral}{"edgelist"}]  = mytree.get\_edgelist(),
    \_[\textcolor{stringliteral}{"postorder"}] = mytree.get\_postorder(),
    \_[\textcolor{stringliteral}{"dag"}]       = mytree.is\_dag()
  );
  
\}

\textcolor{comment}{/***R}
\textcolor{comment}{# set.seed(8)}
\textcolor{comment}{}
\textcolor{comment}{# Is it working?}
\textcolor{comment}{tree <-}
\textcolor{comment}{  structure(}
\textcolor{comment}{    list(}
\textcolor{comment}{      edge = structure(c(4L, 5L, 5L, 4L, 5L, 1L, 2L,}
\textcolor{comment}{                         3L), .Dim = c(4L, 2L)),}
\textcolor{comment}{      tip.label = c("t3", "t2", "t1"),}
\textcolor{comment}{      edge.length = c(}
\textcolor{comment}{        0.718927504029125,}
\textcolor{comment}{        0.290873386897147,}
\textcolor{comment}{        0.932269813260064,}
\textcolor{comment}{        0.769146954407915}
\textcolor{comment}{      ),}
\textcolor{comment}{      Nnode = 2L}
\textcolor{comment}{    ),}
\textcolor{comment}{    class = "phylo",}
\textcolor{comment}{    order = "cladewise"}
\textcolor{comment}{  )}
\textcolor{comment}{ans <- fancytree(tree$edge[,1] - 1, tree$edge[,2] - 1)}
\textcolor{comment}{}
\textcolor{comment}{# plot(tree, show.node.label = TRUE)}
\textcolor{comment}{# cbind(1:length(unlist(ans[[1]])), unlist(ans[[1]]) + 1)}
\textcolor{comment}{# }
\textcolor{comment}{# # DAGS NO DAGS}
\textcolor{comment}{# invisible(fancytree(tree$edge[,1] - 1, tree$edge[,2] - 1))}
\textcolor{comment}{# E <- tree$edge}
\textcolor{comment}{# E <- rbind(E, rev(E[3,]))}
\textcolor{comment}{# E - 1L}
\textcolor{comment}{# fancytree(E[,1] - 1, E[,2] - 1) # NOT A DAG}
\textcolor{comment}{# }
\textcolor{comment}{# library(igraph)}
\textcolor{comment}{# plot(graph\_from\_edgelist(E), vertex.label = 0:4, vertex.size = 40)}
\textcolor{comment}{}
\textcolor{comment}{*/}
\end{DoxyCodeInclude}
 